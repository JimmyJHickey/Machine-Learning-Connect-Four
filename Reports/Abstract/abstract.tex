\documentclass[]{article}

% Margins
\usepackage[letterpaper, top=1in, bottom=1in, left=1in, right=1in]{geometry}

% For prettier tables
\usepackage{array}

% For pictures
\usepackage{graphicx}

% For units
\usepackage{siunitx}

% For enumerates
\usepackage{enumitem}

% For code
\usepackage{courier}
\usepackage{listings}
\lstset{ mathescape }
\lstset{basicstyle=\ttfamily\footnotesize,breaklines=true}

% Change Font
\usepackage[sfdefault]{roboto}  %% Option 'sfdefault' only if the base font of the document is to be sans serif
\usepackage[T1]{fontenc}

% For Header
\usepackage[english]{babel}
\usepackage[utf8]{inputenc}
\usepackage{fancyhdr}

% Make Header
\pagestyle{fancy}
\rhead{Jimmy Hickey}
\lhead{CS 405}

% Page numbers
\pagenumbering{arabic}

% Title
\title{Computers \textit{Attempting} to Kick Human Butt at Connect 4}
\author{Ben Andrews \& Jimmy Hickey}

\begin{document}
	\maketitle
\begin{abstract}
	Connect 4 is a two player, adversarial game in which the players take turns placing pieces on a board. The first player connect four pieces in a line wins. Artificial intelligence scholars became interested in this game and it has since been mathematically solved, that is there is always a definite correct move. For a user, however, playing against a machine that always makes the perfect move and thus always does not create a meaningful experience. Thus, there is a need for a computer player that is good, but not perfect. Through supervised learning algorithms, a machine can be taught to play this game well, but with some inherent flaws due to the stochastic nature of supervised learning. Applying these methods offers a way to generate computer players that can play at different skill levels, offering a purposeful experience to any human player.
\end{abstract}
\end{document}
