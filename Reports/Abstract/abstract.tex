\documentclass[12pt]{article}

% Margins
\usepackage[letterpaper, top=1in, bottom=1in, left=0.75in, right=0.75in]{geometry}

% For prettier tables
\usepackage{array}

% For pictures
\usepackage{graphicx}

% For units
\usepackage{siunitx}

% For enumerates
\usepackage{enumitem}

% For code
\usepackage{courier}
\usepackage{listings}
\lstset{ mathescape }
\lstset{basicstyle=\ttfamily\footnotesize,breaklines=true}

% Change Font
\usepackage[sfdefault]{roboto}  %% Option 'sfdefault' only if the base font of the document is to be sans serif
\usepackage[T1]{fontenc}

% For Header
\usepackage[english]{babel}
\usepackage[utf8]{inputenc}
\usepackage{fancyhdr}

\usepackage{fontawesome}
\usepackage{hyperref}
\usepackage{setspace}

% Page numbers
\pagenumbering{arabic}

% Title
\title{Computers Playing Connect Four at Any Skill Level}
\author{Ben Andrews \& Jimmy Hickey}

\begin{document}
	\maketitle
		\doublespacing
\begin{abstract}

	Connect 4 is a two player, adversarial game in which the players take turns placing pieces on a board; the first player to connect four pieces in a line wins. It has since been mathematically solved, that is there is always a definite correct move. For a user, however, playing against a machine that always makes the perfect move does not create a meaningful experience. Thus, there is a need for a computer player that is good, but not perfect. 
	
	We have implemented a minimax algorithm to provide this service. It can be adjusted to make better or worse decisions; however, this process takes a lot of time. Each move can take minutes to make for higher level minimax players. This is wait is unacceptable for any user, so instead we generated data using the minimax algorithm and used it to train a neural network.
Through supervised learning algorithms and our data, we taught a machine to play Connect Four amply, but with some inherent flaws due to the stochastic nature of the network training. Applying these methods introduces a procedure to generate computer players that can compete at different skill levels with quick response times, offering an enjoyable experience to any user.
\end{abstract}
The codebase and instructions to play against our machine can be found here: \\
\faGithub \ \  \href{https://github.com/JimmyJHickey/Machine-Learning-Connect-Four}{https://github.com/JimmyJHickey/Machine-Learning-Connect-Four}


\end{document}
