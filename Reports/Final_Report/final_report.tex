\documentclass[12pt]{article}

% Margins
\usepackage[letterpaper, top=1in, bottom=1in, left=1in, right=1in]{geometry}

% For prettier tables
\usepackage{array}

% For pictures
\usepackage{graphicx}

% For units
\usepackage{siunitx}

% For enumerates
\usepackage{enumitem}

% For code
\usepackage{courier}
\usepackage{listings}
\lstset{ mathescape }
\lstset{basicstyle=\ttfamily\footnotesize,breaklines=true}

% Clickable Table of Contents
\usepackage{color}   %May be necessary if you want to color links
\usepackage{hyperref}
\hypersetup{
	colorlinks=true, %set true if you want colored links
	linktoc=all,     %set to all if you want both sections and subsections linked
	linkcolor=black,  %choose some color if you want links to stand out
}

% Change Font
\usepackage[sfdefault]{roboto}  %% Option 'sfdefault' only if the base font of the document is to be sans serif
\usepackage[T1]{fontenc}

% For Header
\usepackage[english]{babel}
\usepackage[utf8]{inputenc}
\usepackage{fancyhdr}

% Make Header
\pagestyle{fancy}
\rhead{Andrews, Hickey}
\lhead{CS 467}

% Page numbers
\pagenumbering{arabic}

% Title
\title{Learning to Play Connect 4}
\author{Ben Andrews, Jimmy Hickey}

\begin{document}
	\maketitle
	\tableofcontents
	\clearpage
	
\section{Introduction}
Connect 4 is a popular two player board game. It is an adversarial, turn-based game with rules that children can comprehend. The game is simple to understand, but requires decision making at each turn. This makes it a perfect subject for machine learning. WIth a small rule set, easily defined win conditions, and slow, turn based play a computer can be taught to play the game with some strategy.

\section{Problem Definition}
Our game board is a 7x6 matrix. When it is a player's turn, they choose a column to play a piece. They piece then falls to the lowest opening in that column; the players then alternate turn. This continues until one player connects 4 pieces in a row horizontally, vertically, or diagonally.

\section{Project Goals}
Our initial intents were to teach both a supervised and unsupervised neural network to play the game. We then planned to pit them humans and each other. If trained on the same data, this would offer some insight into which method worked best for this type of problem.

However, our goals changed during the design phase. We decided to take a more data driven approach. We devised a system to generate data that could easily teach a supervised network to play the game.

Though we cut out our unsupervised network, we still wanted to compare techniques. Instead, we tra

\section{Model Formation}
\subsection{Choosing a Model}

\subsection{Minimax}

\section{Implementation}
\subsection{Making the Game}

\subsection{Data Generation}

\subsection{Scikit-learn MLPClassifier}

\section{Computational Study}
\subsection{Performance}

\subsection{Improvements}

\section{Conclusion}

\section{Sources}

\end{document}
